% Define document class
\documentclass[twocolumn]{aastex631}
\usepackage{showyourwork}

% Begin!
\begin{document}

% Title
\title{Refined Flare Rates for Stars Younger than 250\,Myr}

% Author list
\author{Adina~D.~ Feinstein}
\altaffiliation{NHFP Sagan Fellow}
\affiliation{Laboratory for Atmospheric and Space Physics, University of Colorado
Boulder, UCB 600, Boulder, CO 80309}

% Abstract with filler text
\begin{abstract}
    Photometric exoplanet detection missions have revolutionized our understanding
    of stellar activity. Stellar flares are short-duration high-energy radiation
    outbursts which can be readily seen across young (< 800\,Myr) stars and even
    in high counts on mature M dwarfs. Here, we present the results of measuring
    stellar flare rates for very young stars, with ages $4.5 - 200$\,Myr right
    after they formed to understand the early evolution of magnetic activity. We
    have selected a sample of 5,700 stars which belong to 26 associations,
    nearby young moving groups, or clusters. We identified $\sim 29,000$ flares
    using the \texttt{stella} machine-learning algorithm. We measured the flare-
    frequency distribution slopes as a function of age and stellar effective
    temperature.
\end{abstract}

% Main body with filler text
\section{Introduction}
\label{sec:intro}

Lorem ipsum dolor sit amet, consectetuer adipiscing elit.
Ut purus elit, vestibulum ut, placerat ac, adipiscing vitae, felis.
Curabitur dictum gravida mauris, consectetuer id, vulputate a, magna.
Donec vehicula augue eu neque, morbi tristique senectus et netus et.
Mauris ut leo, cras viverra metus rhoncus sem, nulla et lectus vestibulum.
Phasellus eu tellus sit amet tortor gravida placerat.
Integer sapien est, iaculis in, pretium quis, viverra ac, nunc.
Praesent eget sem vel leo ultrices bibendum.
Aenean faucibus, morbi dolor nulla, malesuada eu, pulvinar at, mollis ac.
Curabitur auctor semper nulla donec varius orci eget risus.
Duis nibh mi, congue eu, accumsan eleifend, sagittis quis, diam.
Duis eget orci sit amet orci dignissim rutrum.

This paper is presented as follows.

\section{Methodology}
\label{sec:methods}

\subsection{Sample Selection}



We aim to measure the evolution in flare rates as a function of stellar age for
stars with $1 \leq t_\textrm{age} \leq 250$\,Myr. We used the MOCA Data Base
(Gagné et al. in prep.) to identify nearby young moving groups, associations,
and open clusters with known ages which satisfied our criteria. We identified
unique 26~groups from which we created our sample list. We considered all targets
that are either bona fide members, high-likelihood candidate members, or candidate members.
This resulted in a catalog of 30,889 stars across 26 associations. We summarize
the sample and quote adopted ages for each association
in Table~\ref{tab:sample}.

\begin{deluxetable}{l r r r r r}[!ht]
\tabletypesize{\footnotesize}
\tablecaption{Adopted Ages of each Young Stellar Population and Number of
Stars per Group \label{tab:sample}}
\tablehead{
\colhead{Population} & \colhead{Age [Myr]} & \colhead{N$_\textrm{stars}$} &
\colhead{Ref.}
}
\startdata
AB Doradus  &  133$_{-20}^{+15}$  &  81  & 1 \\
Blanco 1 & 137.1$^{+7.0}_{-33}$ & 301 & 8\\
Carina  &  45  & 91  & 2\\
Carina-Musca & 32 & 22 & 5\\
Chamaeleon  &  5  &  366  & 3 \\
Columba  &  42  &  126  & 2\\
Greater Taurus Subgroup 5 & 8.5 & 56 & 5\\
Greater Taurus Subgroup 8 & 4.5 & 114 & 5\\
Lower Centaurus Crux &  15  & 590  & 4\\
MELANGE-1 & 250$_{-70}^{+50}$ & 22 & 10\\
Octans  &  35$\pm 5$ &   98  & 6 \\
Pisces Eridanis  &  120  &   176  & 7 \\
Pleiades & 127.4$^{+6.3}_{-10}$ & 1304 & 8\\
$\alpha$ Persei  &  79$_{-2.3}^{+1.5}$  &  467  & 8 \\
IC 2602 corona &  52.5$_{-3.7}^{+2.2}$  & 10  & 8 \\
IC 2602 system &  52.5$_{-3.7}^{+2.2}$  & 138  & 8 \\
NGC 2451A & 48.5 & 44 & 5\\
Oh 59 & 162.2 & 50 & 9\\
Platais 9 & 50 & 99 & 11\\
RSG2 & 126 & 87 & 12\\
Theia 301  &  195  &  372  & 9\\
Theia 95  &  30.2  &  191  & 9\\
TW Hydrae  &  10  &  24  & 2\\
Upper Centaurus  Lupus &  16  &  464  & 4 \\
Upper Scorpius  &  10  &  80  & 4 \\
Vela-CG4  &  33.7  &  352  & 5
\enddata
\tablecomments{Age references: (1) \citet{gagne18_abdmg}; (2) \citet{Bell15};
(3) \citet{luhman07}; (4) \citet{pecaut16}; (5) \citet{kerr21};
(6) \citet{murphy15}; (7) \citet{curtis19}; (8) \citet{galindo22};
(9) \citet{kounkel20}; (10) \citet{tofflemeire21}; (11) \citet{tarricq21};
(12) \citet{roser16}; }
\end{deluxetable}

\subsection{TESS Light Curves}

To ensure we were able to resolve and accurately measure the energies of flares
on these stars \citep{howard22}, we used the 2-minute data from the TESS mission.
We crossmatched our sample with the TESS Input Catalog (TIC) based on their RA and Dec.
We considered a star to be in the catalog if the distance between the target and
the nearest TIC was within $1"$. This resulted in a final sample of 5,725 unique targets
which have been observed at 2-minute cadence between Sector 1 and Sector 67, the
latest available sector upon this analysis. Given the proximity of many targets towards the
northern and southern ecliptic poles (Figure~\ref{fig:sample}), many targets were observed
over multiple sectors. We downloaded all available data, yielding a total of 17,964
light curves.

\begin{figure}[ht!]
    \script{sample.py}
    \begin{centering}
        \includegraphics[width=\linewidth]{figures/sample.pdf}
        \caption{
            Distribution of our selected sample across the sky and colored by the
            adopted age of the association (see Table~\ref{tab:sample}). The all-sky
            coverage by TESS has unlocked new populations of stars to observe.
            We take advantage of this observing strategy to measure flare rates
            across 26 different nearby young moving groups, clusters, and associations.
        }
        \label{fig:sample}
    \end{centering}
\end{figure}


\subsection{Flare Identification}

We followed the machine learning flare-identification methods presented in
\cite{feinstein20a}. This method relies on the fact that all flare events
have similar time-dependent morphologies generally described as a sharp rise
and an exponential decay. This method uses a convolutional neural network (CNN),
\texttt{stella} \citep{feinstein20b}, trained on by-eye validated flare events from
TESS Sectors 1 and 2 to identify flare events in TESS 2-minute data
\citep{guenther19_flares, feinstein20a}. The benefits of the CNN is that it is
insensitive to the stellar baseline, since it is trained to look only for the
flare morphology. This means that the peaks of rotational modulation driven by
stellar heterogeneities, which is readily seen in the light curves of young stars,
are not accidentally identified as flares. In this way, we are able to
build a sample of flares which is unbiased towards low-amplitude/low-energy flares,
which are often not identified in traditional sigma-outlier identification
methods \citep{chang15}.

The \texttt{stella} CNN models take the light curve (time, flux, flux error)
as an input and returns an array with values of [0,1], which are treated as
the probability a data point is (1) or is not (0) part of a flare. We ran all
light curves through 10 \texttt{stella} models and averaged the output, taking
this as the final prediction per observation. The \texttt{stella} code uses the
predictions per data point to group together points, identifying those as single
flare events. It then calculates the probability of the entire flare event as
the average of the probabilities assigned to all the data points during the flare.


\begin{figure}[ht!]
    \script{flare_distribution.py}
    \begin{centering}
        \includegraphics[width=\linewidth]{figures/flare_distribution.pdf}
        \caption{
            High level summary of the demographics of flares in our sample. Top:
            The number of flares identified compared to the number of stars in
            each nearby young moving group, cluster, or association. A one-to-one
            relationship is expected. Middle: The distribution measured TESS energies
            and equivalent durations of flares in our sample, colored by the probability
            of the flare as identified with \texttt{steela}. Bottom: Same as the
            middle plot, except colored by the $T_\textrm{eff}$ [K] of the star.
            We limit our sample to stars with $T_\textrm{eff} \leq 6000$\,K.
        }
        \label{fig:flare_distribution}
    \end{centering}
\end{figure}

\subsection{Modeling Flare Properties}

We used the analytical flare model in
\cite{tovar22}\footnote{\url{https://github.com/lupitatovar/Llamaradas-Estelares}}
to fit the flares in our sample. This model builds upon the model presented
in \cite{davenport14}, and includes a convolution of a Gaussian with a
double exponential model. The analytical model accounts for the amplitude,
heating timescale, rapid cooling phase timescale, and slow cooling phase
timescale of flares. Using this model, we used a non-linear least squares
optimization to fit for the time of the flare peak ($t_\textrm{peak}$),
full width at half maximum (FWHM), and the amplitude ($A$), of each flare
in the sample. We combine the model with a second-order polynomial fit to
a 10-hour baseline before and after the flare to account for any rotational
modulation.

From these fits, we performed a series of quality checks to ensure the flares
in our sample are real. First, we removed any flares with best-fit FWHM $ < 0$
and $A < 0$. Second, we removed any flares with a large error on the amplitude,
$\sigma_A > 10$ and flares where $\sigma_A > A$. Third, we removed flares where
$A < 2 \times \textrm{RMS}$, where the RMS is calculated from the 10~hour baseline
around the best-fit $t_\textrm{peak}$. These cuts are indicative of a poorly fit
flare model due to the flare being overwhelmed by noise in the surrounding baseline
light curve. We choose to include flares with a probability $P \geq 75\%$ of being
a true flare. After these checks, we have a flare sample of 29,128~flares originating
from 3,983 stars (Figure~\ref{fig:flare_distribution}). 83.6\% of the flares have
probabilities of $P \geq 90\%$ of being true; 64.9\% of the flares have probabilities
of $P \geq 99\%$ of being true.


\subsection{Measuring Rotation Periods}

\cite{seligman22} demonstrated that stars with low Rossby numbers $R_0 < 0.13$ have
shallower flare frequency distribution slopes, indicative of more high energy flares
originating from these sources. In addition to understanding flare statistics across
young ages, we aim to expand this sample by measuring the rotation periods, $P_\textrm{rot}$,
for the stars in our sample. To do this, we used \texttt{michael}\footnote{\url{https://github.com/ojhall94/michael}},
an open-source Python package that robustly measures $P_\textrm{rot}$ using a combination
traditional Lomb-Scargle periodograms and wavelet transformations (Hall et al. submitted).
\texttt{michael} measures $P_\textrm{rot}$ using the \texttt{eleanor} package, which
extracts light curves from the TESS Full-Frame Images \citep[FFIs;][]{feinstein19}.
We ran \texttt{michael} on all stars from which flares were identified. The estimated
$P_\textrm{rot}$ were vetted by-eye, from the \texttt{michael} diagnostic plots. In
total, we robustly measured \textcolor{red}{n} $P_\textrm{rot}$ across our sample of
3,983 stars.


\section{Results}\label{sec:results}

We analyze our new flare sample from three perspectives. First, we perform the standard
FFD fitting of a power-law to the distribution of flare energies. Second, we fit the FFD with the
prescription in \cite{gershberg72}, which fits for both the FFD slope and y-intercept.
Third, we fit a truncated power-las to the distribution of flare amplitudes, to determine
if there is a correlation between $R_0$ and flare distributions.

The number of stars, and consequently flares, per each association varied greatly,
due to the limited number of stars observed at TESS 2-minute cadence. Therefore, instead
of measuring FFD properties as a function of association, we opted to group stars
by effective temperature, $T_\textrm{eff}$, and average adopted association age. We
grouped stars in the following $T_\textrm{eff}$ space: M-stars below the fully
convective boundary ($T_\textrm{eff} = 2300 - 3400$\,K), early type M-stars
($T_\textrm{eff} = 3400 - 3850$\,K), late K-stars ($T_\textrm{eff} = 3850 - 4440$\,K),
early K-stars ($T_\textrm{eff} = 4440 - 5270$\,K), and G-stars ($T_\textrm{eff} = 5270 - 5930$\,K).
We did not include any stars hotter than $T_\textrm{eff} > 5930$\,K, as these stars
are dominated by noise in the TESS observations. Additionally, we grouped stars
in the following age space: $4-10$\,Myr, $10-20$\,Myr, $20-40$\,Myr, $40-50$\,Myr,
$70-80$\,Myr, $120-150$\,Myr, and $150-300$\,Myr. We note that there is a gap in
age from $50-70$\,Myr, which could be expanded with the identification of more
associations in this age range. However, for the purposes of this work, we do not
include additional sources which may fall in this age range.


\subsection{Standard Power-Law Fits}\label{subsec:stats}

From the $T_\textrm{eff}$ and age bins described above, we binned the flares in each
subgroup and fit their FFD slopes, approximated as a power-law. Flares were binned
into 25 bins in log-space from $10^{27} - 10^{35}$\,erg. We fit the FFDs from the
energy bin with the maximum flare rate and energies higher than that. We opt to do
this as bins of lower energies may be incomplete, and the turnover in the FFD cannot
be accurately modeled as a power-law.

We fit the FFD using the MCMC method implemented in \texttt{emcee} \citep{goodman10, emcee}
and fit for the slope, $\alpha$, y-intercept, $b$, and an additional noise term, $f$, which
accounts for an underestimation of the errors on each bin. We initialized the MCMC fit with
300 walkers and ran our fit over 5000 steps. Upon visual inspection, we discarded the
first 100 steps; onwards the steps were fully burned-in. The full FFDs are presented
in Figure~\ref{fig:simple_ffd_all}, along with 100 samples from the MCMC fit. The measured
FFD slopes, $\alpha$ are presented in Figure~\ref{fig:mcmc_results}. We approximate the
error on the slope as the lower 16\textsuperscript{th} and upper 84\textsuperscript{th}
percentiles from the MCMC fit.


\begin{figure*}[ht!]
    \script{mcmc_results.py}
    \begin{centering}
        \includegraphics[width=3\textwidth]{figures/mcmc_results.pdf}
        \caption{
            Measured flare-frequency distribution slopes, $\alpha$, as a function
            of stellar effective temperature, $T_\textrm{eff}$ and age.
        }
        \label{fig:mcmc_results}
    \end{centering}
\end{figure*}


\subsection{Fitting for $\alpha$ and $\beta$}


\subsection{Truncated Power-Law Fits}

We follow the prescription presented in \cite{seligman22}. Namely, we fit a truncated
power-law distribution of the form

\begin{equation}
  dp/dA \propto A^{-\alpha_T} e^{-A/A_*}
\end{equation}

where $A$ is the amplitude of the flare, $A_*$ is a flare amplitude cutoff parameter
and $\alpha_T$ is the slope, rather than
$\alpha$. We fit the slopes using the MCMC method implemented in \texttt{emcee}
\citep{goodman10, emcee}, using the log-likelihood function in \cite{seligman22}.
We fit for $A_*$ and $\alpha_T$. We initialized the MCMC fit with textcolor{red}{n}
walkers and evaluated the fit over textcolor{red}{n} steps. The first textcolor{red}{n}
steps were discarded upon visual inspection. The results are presented in Figure~\ref{fig:trunc}.

\section{Discussion}\label{sec:discuss}

\subsection{Applications to ULLYSES}

Measuring Far-Ultraviolet (FUV) total flux output and flare rates are
essential for understanding the contributions of flares to exoplanet atmospheric
mass-loss and disequilibrium chemistry. An equivalent, but significantly smaller sample
to the stars studied here are the those observed as part of the \textit{Hubble Space Telescope}
(HST) Ultraviolet Legacy Library of Young Stars as Essential Standards (ULLYSES)
program. This survey observed 71 K- to M-type T Tauri stars in nine young
galactic associations.

We can extend the results presented here to estimate how many flares may be present
in these important FUV observations.

Additionally, we analyzed the TESS light curves for all of the ULLYSES targets, when
available at 2-minute cadence. We identified a total of \textcolor{red}{n} flares.
This is inconsistent with the overall flare statistics identified in this work, and is
overall unexpected for young stars which are known to be more magnetically active
\citep{ilin20, feinstein20b}. This could suggest that there is some bias in the way
the ULLYSES sample was constructed, and that it is unexpected to find any flares in the
FUV data. Alternatively, it could be that the ongoing accretion luminosity results
in a brighter host star, such that the flares we normally would see are drowned
out by the increased baseline luminosity. We do not include this flare in our
overall statistical analysis of TESS flare rates for young stars.


\section{Conclusions}\label{sec:conclusions}


This work made use of the open-source package, \textcolor{red}{\textit{showyourwork!}}
\citep{luger2021}, which promotes reproducible publications.

\appendix
\restartappendixnumbering

\section{Supplemental Material}\label{appendix:supp_ffds}

Miscellaneous figures and such that people might want but I don't need to show

\begin{figure*}[ht!]
    \script{simple_ffd_all.py}
    \begin{centering}
        \includegraphics[width=\textwidth]{figures/simple_ffd_all.pdf}
        \caption{
            Flare frequency distributions (FFDs) for subgroups of stars, clustered
            by age and effective temperature, $T_\textrm{eff}$. Flares were binned
            into 25 bins in log-space from $10^{27} - 10^{35}$\,erg. We fit the FFD
            from the turn-over in the binned flares, likely a result of very low-energy
            flares being missed in our flare-detection algorithm. The bins used to fit
            the FFD are shown in black, while all bins are shown in gray. We ran an MCMC
            fit to these distributions with a simple power law; 100 random samples from these
            fits are over-plotted in orange. We fit distributions with $> 3$ bins.
            The best-fit slopes from these fits are presented in Figure~\ref{fig:mcmc_results}.
        }
        \label{fig:simple_ffd_all}
    \end{centering}
\end{figure*}


\bibliography{bib}

\end{document}
